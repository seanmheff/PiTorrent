\chapter{Design and Implementation} 

\label{Chapter3} 

\lhead{Chapter 3. \emph{Design and Implementation}} 

%----------------------------------------------------------------------------------------

\section{Proof of concept}
As I have outlined in chapter 1, my project should work by utilising UNIX sockets for communication between Node.js and rTorrent. Communication between the back-end and rTorrent traditionally used a web-server with SCGI plugins like so: 

\begin{figure}[h!]
  \centering
    \includegraphics[scale=0.75]{Figures/ruTorrent.png}
  \caption{ruTorrent setup}
\end{figure}

Users would make HTTP requests to the ruTorrent back-end to get torrent information. The back-end then would then have to route requests to rTorrent via the SCGI web-server plugin. This is not ideal, as all XML-RPC requests between ruTorrent and rTorrent are routed through the SCGI plugin, and this method of communication is open to abuse. It may be possible for malicious individuals to send commands to rTorrent directly, by bypassing ruTorrent on an unsecured setup. An attack is possible as the SCGI interface runs off a port on the web-server, a port which may be accessible to unauthorized users on the network. My project aims to remove the SCGI dependency by spoofing SCGI requests in my Node.js back-end and sending these requests directly to rTorrent's XML-RPC socket like so:

\begin{figure}[h!]
  \centering
    \includegraphics[scale=0.8]{Figures/PiTorrent.png}
  \caption{PiTorrent setup}
\end{figure}

This has the added benefit of being easier to set up - there is no need to install and configure surplus technologies.


\subsection{XML-RPC Message Format}
In order to talk about spoofing SCGI requests to do away with the SCGI plugin, I will need explain the format of a typical XML-RPC message. XML-RPC is a simple specification with the purpose of calling remote functions over a network. The format is built on top of HTTP. A XML-RPC message is essentially just a HTTP POST request with a specific XML format as the request body. Here is an example XML-RPC request:

\vspace{20px}
\begin{lstlisting}[caption=A basic XML-RPC request, language=XML]
  POST /RPC2 HTTP/1.0
  User-Agent: Frontier/5.1.2 (WinNT)
  Host: localhost
  Content-Type: text/xml
  Content-length: 181

  <?xml version="1.0"?>
  <methodCall>
    <methodName>getStateName</methodName>
    <params>
      <param>
        <value><i4>41</i4></value>
      </param>
    </params>
  </methodCall>
\end{lstlisting}

The header contains typical HTTP header information. The body contains an XML document with specific node names. The values contained in these nodes must specify the name of the remote method we wish to call, and any parameters we wish to pass to the method. Parameter values can be any of the following types:

\begin{itemize}
	\item <i4> or <int>
	\item <boolean>
	\item <string>
	\item <double>
	\item <dateTime>
	\item <base64>
	\item <struct>
\end{itemize}

In the example above, we are passing an integer with the value 41 to the method "getStateName".

\subsection{SCGI Message Format}
SCGI is a wrapper around XML-RPC. As incoming remote XML-RPC messages are received, our SCGI plugin will wrap the entire XML-RPC message with a header and send the newly wrapped message to the RPC socket. The layout of a SCGI header can be described with the following thing:

\vspace{20px}
\begin{lstlisting}[caption=Layout of a SCGI header]
  headers ::= header*
  header ::= name NUL value NUL
  name ::= notnull+                
  value ::= notnull*
  notnull ::= <01> | <02> | <03> | ... | <ff>
  NUL = <00>
\end{lstlisting}

The first header must have the name "CONTENT\_LENGTH", with the corresponding value containing the length of the body. A "SCGI" header must also be present, with a value of "1". SCGI requests take the form of [length of SCGI header]":"[SCGI header]","[data]. Here is the SCGI request created by wrapping the XML-RPC request in Figure 3.1. 

\vspace{20px}
\begin{lstlisting}[caption=Layout of a SCGI request, language=XML]
  26:
     CONTENT_LENGTH<00>181<00>
     SCGI<00>1<00>
  ,
  <?xml version="1.0"?>
  <methodCall>
    <methodName>getStateName</methodName>
    <params>
      <param>
        <value><i4>41</i4></value>
      </param>
    </params>
  </methodCall>
\end{lstlisting}


After researching XML-RPC and SCGI, I determined that I could indeed spoof SCGI requests within Node.js by replicating the SCGI protocol, and sending these requests to rTorrent's RPC interface over a UNIX socket. Here is some Node.js code that I wrote as a proof of concept of creating spoofed SCGI requests:

\vspace{20px}
\begin{lstlisting}[caption=Layout of a SCGI request, language=JavaScript]
  function formatRequest(xml) {  
    /* Build the header of the request */
    var header = "";
    header += "CONTENT_LENGTH\000";
    header += xml.length;
    header += "\000";
    header += "SCGI\0001\000";
  
    /* Build the request */
    var request = "";
    request += header.length;
    request += ":";
    request += header;
    request += ",";
    request += xml;
  
    return request;
  }
\end{lstlisting}

I tried sending some spoofed SCGI requests to rTorrent's RPC socket using Node.js and listening for responses (Node.js code to achieve this contained in Figure 2.1), and low and behold - it worked! I now knew that my project was viable and I could start building it.


\section{Structuring the Project}
Due to the Raspberry Pi's limited resources, I wanted the Node.js back-end to do as little computation as possible. One way to achieve this is by using a web development technique know as "client-side rendering". With client-side rendering, when you request a web-page from a server, the server sends you a static HTML page, along with any CSS or JavaScript code that the HTML page requires. The JavaScript code then makes another request to the server for any dynamic data needed to render the page (in our case - torrent information). Once the JavaScript gets the response, it modifies the original HTML page to include the data it received from the server's Application Programming Interface (API), and renders the page.

With this technique, the Pi does not need to generate dynamic HTML pages - instead it pushes this computation to the client. We also have the added benefit of solving another problem - how to continually send live torrent information to the client. In order to perform client-side rendering, we need to create server API's where live torrent information can be consumed by our JavaScript. These API's can then be polled at set intervals by our JavaScript to get up-to-date torrent information.

Here is the general structure that my project should take:

\vspace{10px}
\begin{figure}[h!]
  \centering
    \includegraphics[scale=1]{Figures/Structure2.png}
  \caption{Project structure}
\end{figure}


\section{Back-end development}
Our Node.js back-end has two distinct jobs:

\begin{enumerate}
	\item Serving static HTML pages, JavaScript and CSS files
	\item Providing API's that provide up-to-date rTorrent information.
\end{enumerate}

\subsection{Serving static files with Express}
Job number 1 is trivial with Node.js and the Express framework. When I talked about Express in Chapter 2, I mentioned that we could use it to accomplish a lot of commonly used tasks. Serving static files is one of these tasks. We can supply Express with a directory and Express will recursively serve all the files within this directory. Here is all the code required to serve content in the "public" folder on port 80:

\vspace{20px}
\begin{lstlisting}[caption=Express serving static files, language=JavaScript]
  var express = require('express')
    , http = require('http')
    , path = require('path')
    , app = express()
    , port = 80;
 
  // Tell our Express app to use the 'static' middleware module and
  // specify a directory where static content resides
  app.use(express.static(path.join(__dirname, 'public')));

  // Create a web server on port 80
  http.createServer(app).listen(port, function() {
    console.log('Express server listening on port ' + port);
  });
\end{lstlisting}

Without Express we would have to manually parse the URL of incoming requests to see what file they wish to download, check if the file exists, and if so, read the file contents into a buffer and pipe the buffer contents to a HTTP response. If the file does not exist, we would have to send a 404 message. Its clear to see why using a framework like Express is a no-brainer.  

\subsection{Creating rTorrent API's}
Job number 2 on our list of server tasks is to provide a set of API's where rTorrent information can be accessed from. We should also be able to "control" rTorrent from these API's, e.g. we should be able to stop a torrent by sending a HTTP POST request to a certain API.

These API's will act as the connector between our front-end and rTorrent. If the front-end wants to see all the torrents that rTorrent is currently managing - the front-end JavaScript should send a HTTP request to a specific URL, say "http://localhost/torrents" (if the Node.js server was running locally). Once Node.js gets a request to this URL, Node should call some XML-RPC method on rTorrent's socket to get a list of all the torrents. rTorrent will respond with an XML document, which our Node server should parse, and send a human readable JSON object to the front-end.

I need to convert the rTorrent XML response to JSON as our front-end JavaScript library, AngularJS can only work with JSON objects. I had two options to convert this data - either convert it on the server and send JSON data to the front-end over HTTP, or send XML to the front-end, and let the front-end JavaScript convert it to JSON. I choose to convert it on the server as JSON is more compact that XML, and by having the server send JSON data over the network will reduce bandwidth usage.

Here I will go through all the steps in the front-end getting live torrent data from the server:

Firstly the following front-end code will call a server API ("/torrents"):
\vspace{10px}
\begin{lstlisting}[caption=AngularJS performing a HTTP GET, language=JavaScript]
  $http.get(document.location.origin + '/torrents').success(function(torrents) {
    $scope.torrentResults = torrents;
  });
\end{lstlisting}


When Node.js gets a request for the "/torrents" URL it will create a specific XML-RPC request. This XML-RPC call attempts to get the hash, name and size of all the torrents in the rTorrent "main" view (rTorrent views include main, seeding, leeching, etc.):
\vspace{10px}
\begin{lstlisting}[caption=Node.js generated XML-RPC request, language=XML]
  <?xml version="1.0" encoding="UTF-8"?>
  <methodCall>
    <methodName>d.multicall</methodName>
    <params>
      <param>
        <value>
          <string>main</string>
        </value>
      </param>
      <param>
        <value>
          <string>d.get_hash=</string>
        </value>
      </param>
      <param>
        <value>
          <string>d.get_name=</string>
        </value>
      </param>
      <param>
        <value>
          <string>d.get_size_bytes=</string>
        </value>
      </param>
    </params>
  </methodCall>
\end{lstlisting}

A SCGI header is added to the request, and it is sent to rTorrent via a UNIX socket. A XML response is received. When the SCGI header is removed from the response, we get the following:
\vspace{10px}
\begin{lstlisting}[caption=XML-RPC response, language=XML]
  <?xml version="1.0" encoding="UTF-8"?>
  <methodResponse>
    <params>
      <param>
        <value>
          <array>
            <data>
              <value>
                <array>
                  <data>
                    <value>
                      <string>BB1D8144644F1F2CA9C2CDE60B3FB1B8A904A3C9</string>
                    </value>
                    <value>
                      <string>linuxmint-16-mate-dvd-32bit.iso</string>
                    </value>
                    <value>
                      <i8>1289748480</i8>
                    </value>
                  </data>
                </array>
              </value>
              <value>
                <array>
                  <data>
                    <value>
                      <string>5741F87F027608169473EE68D2EA7DDBBFA4EA4A</string>
                    </value>
                    <value>
                      <string>debian-7.4.0-i386-CD-1.iso</string>
                    </value>
                    <value>
                      <i8>679477248</i8>
                    </value>
                  </data>
                </array>
              </value>
            </data>
          </array>
        </value>
      </param>
    </params>
  </methodResponse>
\end{lstlisting}

You can see why I chose to parse this data on the server and not send the raw XML to the client. The XML is very verbose and by converting it to JSON, I can make significant bandwidth savings. Once I have the XML response, I make the conversion to JSON and send the JSON object to the front-end over HTTP. Here is what that HTTP response would contain:
\vspace{10px}
\begin{lstlisting}[caption=XML-RPC response converted to JSON, language=JavaScript]
  {
    "torrents": [
      {
        "hash": "BB1D8144644F1F2CA9C2CDE60B3FB1B8A904A3C9",
        "name": "linuxmint-16-mate-dvd-32bit.iso",
        "size": "1289748480",
      },
      {
        "hash": "5741F87F027608169473EE68D2EA7DDBBFA4EA4A",
        "name": "debian-7.4.0-i386-CD-1.iso",
        "size": "679477248",
      }
    ]
  }
\end{lstlisting}

I created a number of API's that the front-end could call. Here are just some of these API's:

\begin{itemize}
	\item "/torrents" - list the names and basic information of all torrents
	\item "/stats" - list some global stats: upload rate, download rate, etc.
	\item "/files/:hash" - list the files for a specific torrent: "hash" is the torrent identifier
	\item "/peers/:hash" - list the peers connected to a specific torrent
	\item "/stop/:hash" - stop a specific torrent
	\item "/start/:hash" - start a specific torrent
\end{itemize}


\section{Front-end development}
Once I have API's to retrieve torrent data from, I need to display this data in a meaningful way. The API data is potentially always changing, so these API's need to be polled at regular intervals so users get up-to-date information about their torrents. Since the main purpose of the front-end is to display torrent data to the user, I choose to use the AngularJS JavaScript framework to help me manage the task of keeping the view up-to-date. 

AngularJS has some neat features. I have already mentioned it's two-way data binding - which I use extensively in this project. This data binding allows me to poll the server API's continually with JavaScript code, and any changes in the data are automatically shown in the view. By using AngularJS I reduced the time taken to code the front-end by a significant amount. Angular takes care of a lot of the boilerplate code that is needed in updating data.

The front-end contains a lot of dynamic data. Anywhere I want to display this data, I insert AngularJS "\{\{data bindings\}\}" in my view markup. I can then use the "\$scope" object in my JavaScript code to associate these bindings with data. Now that I have an idea of how to show the user updated torrent data, I need to decide on the layout of my front-end.

\subsection{Layout of the web-app}
I decided to give the front-end a standard web-app layout:

\vspace{10px}
\begin{figure}[h!]
  \centering
    \includegraphics[scale=0.32]{Figures/Home.png}
  \caption{Front-end layout}
\end{figure}

There is a navigation panel on the left-hand side so that users can easily find there way around the app at all times. This panel is responsive, and will be hidden by default on tablets and other smaller screen devices. 

There are horizontal bars running across both the top and bottom of the app. These bars are fixed to the top and bottom of the screen, so that if you scroll the page up or down, the bars will remain in place. The top bar has a PiTorrent logo on the left-hand side and a dropdown menu on the right-hand side. The dropdown is typical of a dropdown one would expect to see on the top right corner of we a web-app - it contains a link to the settings page, and a logout link. 

The bottom bar has torrent speed and torrent throttle indicators, so that we can easily see what our upload and download speeds are at all times. A user can also click on the throttle indicators and be presented with a "popover" window where they can drag a slider to set upload or download throttle speeds. Here is what that popover looks like:

\vspace{10px}
\begin{figure}[h!]
  \centering
    \includegraphics[scale=0.75]{Figures/Throttle.png}
  \caption{Download throttle configuration}
\end{figure}

The main section of the app will display data to the user. I decided to use a "widget" theme with my data, where different data sections are partitioned across widgets. These widgets are then arranged in a grid format. The grid is collapsible, so if you are viewing the site on a smaller screened device the grid columns merge into one. 

The default page when you log in is the "torrents overview" page. This page shows lists of currently uploading and download torrents (if any), along with graphs of upload and download speeds over the last five minutes. The overview page also has a "System Stats" and a "Add Torrents" widget.

The System Stats widget gives users information about the Raspberry Pi. Users can see uptime, load averages and memory usage, amongst other things. The Add Torrents widget allows users to add torrents to the system. Users can either upload a .torrent file from their computer, or they can remotely fetch a .torrent file by providing a URL. 

Along the top of the main section, there is a set of tabs where users can view torrents in different states. States include: All, Stopped, Started, Seeding, Leeching, Currently Uploading and Currently Downloading. By clicking a tab, users can quickly see all the torrents in that state.

Users can click on a torrent and see detailed information about that torrent. When a user clicks on a torrent AngularJS will intercept the event and change the main section of the view to another view where we can see detailed torrent information. When users are in the "detailed information" view, they have three sections that they can look at: Overview, Peer and Files. 

The overview section displays general information about the torrent (upload/download speeds, amount of data uploaded/downloaded, etc.). The Peers section shows a list of all the peers we are connected to. The list contains information about each of the peers (IP address, the speeds we are uploading/downloading to/from the peer, etc.). Lastly, the Files section shows us a list of all the files in the torrent, and what percentage downloaded each file is. This can be handy for large downloads, where we wish to see if a specific part of the torrent has downloaded. Here is what the overview section looks like, notice the tabs to enter the other sections:

\vspace{10px}
\begin{figure}[h!]
  \centering
    \includegraphics[scale=0.75]{Figures/DetailedInfo.png}
  \caption{Overview section of detailed information view}
\end{figure}

Finally, there is a Settings section in the app. This section allows the user to change the system's settings. When I was coding the back-end, I made sure to include any environment specific parameters in a config file. This config file contains parameters like the location of the rTorrent RPC socket and the directory the back-end should monitor for .torrent files. The settings section of the app enables the user to change these back-end parameters without editing the config file on the server via SSH or FTP.

Here is a screenshot of the settings section:

\vspace{10px}
\begin{figure}[h!]
  \centering
    \includegraphics[scale=0.75]{Figures/Settings.png}
  \caption{The settings section}
\end{figure}











